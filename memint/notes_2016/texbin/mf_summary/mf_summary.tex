\documentclass[a4paper,10pt]{article}
%\documentclass[a4paper,10pt]{scrartcl}

\usepackage[utf8x]{inputenc}
\usepackage{amsmath}
\usepackage{bbold}
\usepackage{graphicx}

\hoffset=-50pt
\textwidth=440pt

\newcommand{\pone}{P(1)}
\newcommand{\pzero}{P(0)}
\newcommand{\py}{P(y)}

\newcommand{\wone}{w(1)}
\newcommand{\wzero}{w(0)}
\newcommand{\wy}{w(y)}

\newcommand{\I}{\mathbb{1}}
\newcommand{\D}{\mathrm{d}}
\newcommand{\e}{\mathrm{e}}
\newcommand{\Or}{\mathcal{O}}
\newcommand{\dL}{\frac{d}{L}}
\newcommand{\OL}{\Or\left(\frac{1}{L}\right)}
\newcommand{\OLL}{\Or\left(\frac{1}{L^2}\right)}

\newcommand{\bra}[1]{\langle #1 \vert}
\newcommand{\ket}[1]{\vert #1 \rangle}
\newcommand{\braket}[2]{\langle #1 \vert #2 \rangle}

\newcommand{\Ai}{\mathrm{Ai}}

\title{Mean Field Thoery}
\author{}
\date{}

\pdfinfo{
  /Title    ()
  /Author   ()
  /Creator  ()
  /Producer ()
  /Subject  ()
  /Keywords ()
}

\begin{document}
\maketitle

\section{Master Equation, Weight Functions, Generating Function}

\begin{figure}[h!]
  \centering
  \includegraphics[width=0.5\textwidth]{mf_schematic}
\end{figure}

\begin{eqnarray}
  \frac{\partial \wy}{\partial t} & = & uw(y-1) \I_{y>0} + \frac{1}{4}w(y+2) + (1-u)[1-\pzero]^L w(y+1) \nonumber \\
				  & - & uw(y) - \frac{1}{4} \wy \I_{y>1} - (1-u) [1-\pzero]^L \wy
\end{eqnarray}
The function $w(y)$ is the statistical weight of having the piece of the interface a distance $y$ from the membrane. The factor $1/4$ comes from the TASEP maximal current $\rho(1-\rho)$ when the density is $1/2$. The factor $[1-\pzero]^L$ describes the probability that all sites have $y>0$.

We define the generating function
\begin{equation}
 G(z) = \sum_{y=1}^{\infty}z^y\wy \;,
\end{equation}
and in the steady state we find
\begin{equation}
  G(z) = \frac{\wone z^2 + (\wone + (1+b)\wzero)z + \wzero}{-a(z-z_-)(z-z_+)} \;,
\end{equation}
where
\begin{equation}
  z_\pm = \frac{ (1+b) \pm \sqrt{ (1+b)^2 + 4a }}{2a}
\end{equation}
and
\begin{equation}
  a = 4u \;, \quad b = 4(1-u)(1-\pzero)^L \;.
\end{equation}
Because
\begin{equation}
  |z_-| < |z_+|
\end{equation}
the pole at $z_-$ is nearer the origin, and will dominate the integral
\begin{equation}
 \wy = \oint \frac{\D z}{2\pi i} \frac{G(z)}{z^y}
\end{equation}
at large $n$, but because 
\begin{equation}
  z_- < 0 
\end{equation}
this means that the distribution would oscillate between positive and negative values as $n$ is increased. Negative values are obviously unphysical, so the term $(z-z_-)$ in the denominator must be cancelled by the numerator. Thus,
\begin{equation}\label{eq:Gz_numerator}
  \wone z^2 + (\wone + (1+b)\wzero)z + \wzero = -a(Az+B)(z-z_-)\;.
\end{equation}
This also gives us the condition
\begin{equation}
  \wone(1+z_-)z_- + \wzero(1+(1+b)z_-) = 0 \;.
\end{equation}
The generating function can now be written
\begin{equation}
  G(z) = \frac{Az + B}{z-z_+} \;,
\end{equation}
which can be expanded in powers of $z$ to give
\begin{equation}
  G(z) = -\frac{B}{z_+} - \left( A + \frac{B}{z_+} \right) \sum_{y=1}^\infty \left( \frac{z}{z_+} \right)^y
\end{equation}
To find $A$ and $B$ we expand \eqref{eq:Gz_numerator}, from which we find
\begin{equation}
  \wone = -aA \;, \quad \wzero = az_-B \;.
\end{equation}
Thus, using $z_+z_- = -1/a$,
\begin{equation}
  G(z) = \wzero - \left( z_+z_-\wone - \wzero \right) \sum_{y=1}^\infty \left( \frac{z}{z_+} \right)^y \;.
\end{equation}
We read off
\begin{equation}
  \wy = \frac{\wzero - z_+z_- \wone}{z_+^y} \;, \quad y > 0\;,
\end{equation}
and use self-consistency at $y=1$ to find
\begin{equation}\label{eq:wy}
  \wy = \frac{\wzero}{(1+z_-)} \frac{1}{z_+^y} \;.
\end{equation}

\subsection{General Form of Probability Interface does not Touch the Membrane}

Instead of inserting $(1-P(0))^L$ in the master equation, we can simply define a generic, unspecified probability $K$ that the interface does not touch the membrane (i.e. that for all interface sites the height $y>0$).

\section{Finding $\py$ Using the Transfer Matrix Approach}

So far the mean field theory doesn't take into account any restrictions in the difference in height between neighbouring sites on the interface. We can do this by defining a transfer matrix which allows us to select only configurations where the difference in heights between neighbouring sites is exactly one (like the Restricted Solid on Solid condition, RSOS). 

We define
\begin{equation}
  T = \begin{pmatrix}
       0      & w(0)   & 0      & 0      & 0      & \cdots \\
       w(1)   & 0      & w(1)   & 0      & 0      &        \\
       0      & w(2)   & 0      & w(2)   & 0      &        \\
       0      & 0      & w(3)   & 0      & w(3)   &        \\
       \vdots &        &        &        &        & \ddots \\
      \end{pmatrix} \;.
\end{equation}
Using \eqref{eq:wy}

\begin{equation}
  T = \frac{\wzero}{(1+z_-)}
      \begin{pmatrix}
       0      & (1+z_-)& 0      & 0      & 0      & \cdots \\
       q      & 0      & q      & 0      & 0      &        \\
       0      & q^2    & 0      & q^2    & 0      &        \\
       0      & 0      & q^3    & 0      & q^3    &        \\
       \vdots &        &        &        &        & \ddots \\
      \end{pmatrix} \;,
\end{equation}
where 
\begin{equation}
  q = z_+^{-1} \;.
\end{equation}
Now, the partition function for  a lattice of size $L$ with periodic boundary conditions and which conforms to the RSOS rule is
\begin{equation}
  Z = \sum_{y=0}^\infty \bra{y} T^L \ket{y} \;,
\end{equation}
where $\ket{y}$ is a (right) basis vector of the space of heights:
\begin{equation}
  \ket{y}_i = \delta_{y,i} \;.
\end{equation}
The probability of an interface site having height $y$ is now 
\begin{equation}
  \py = \frac{\bra{y} T^L \ket{y}}{Z} \;.
\end{equation}
A factor $(\wzero/(1+z_-))^L$ factorises from both top and bottom, so we can redefine $T$:
\begin{equation}\label{eq:T}
  T = \begin{pmatrix}
       0      & (1+z_-)& 0      & 0      & 0      & \cdots \\
       q      & 0      & q      & 0      & 0      &        \\
       0      & q^2    & 0      & q^2    & 0      &        \\
       0      & 0      & q^3    & 0      & q^3    &        \\
       \vdots &        &        &        &        & \ddots \\
      \end{pmatrix} \;.
\end{equation}
The matrix $T$ satisfies an eigenvalue equation
\begin{equation}
  T \ket{\mu} = \mu \ket{\mu} \;,
\end{equation}
where
\begin{equation}
  \ket{\mu} = \sum_{y=0}^\infty \phi^{(\mu)}_y \ket{y} \;.
\end{equation}
Thus
\begin{eqnarray}
  \py &  = & \frac{\bra{y} T^L \sum_\mu \ket{\mu} \braket{\mu}{y}}{\sum_{y'=0}^\infty \bra{y'} T^L \sum_\mu \ket{\mu} \braket{\mu}{y'}} \nonumber \\
      &  = & \frac{ \sum_\mu \mu^L \braket{y}{\mu} \braket{\mu}{y}}{\sum_\mu \mu^L \sum_{y'=0}^\infty \braket{y'}{\mu} \braket{\mu}{y'}}\nonumber \\
      &  = & \frac{\mu^L \braket{y}{\mu} \braket{\mu}{y}}{\sum_{y'=0}^\infty \braket{y'}{\mu} \braket{\mu}{y'}}  \;.
\end{eqnarray}
If we then assume that this sum is dominated by the largest eigenvalue for large $L$, and we abuse our notation a bit and call this largest eigenvalue $\mu$, its corresponding right-eigenvector $\ket{\phi}$ and its corresponding left-eigenvector $\bra{\psi}$, then 

\begin{equation}
  \py  =  \frac{ \braket{y}{\phi} \braket{\psi}{y}}{\sum_{y'=0}^\infty \braket{y'}{\phi} \braket{\psi}{y'}}  \;.
\end{equation}
So, to find $\py$, we just need to find the right and left eigenvectors corresponding to the largest eigenvalue $\mu$ of $T$.

\subsection{Right-Eigenvector Recursion}

By writing the right eigenvector $\ket{\phi}$ as a sum over the basis vectors $\ket{y}$:
\begin{equation}\label{eq:phi_basis}
  \ket{\phi} = \sum_{y=0}^\infty \phi_y \ket{y} \;,
\end{equation}
We find that  coefficients $\phi_y$ satisfy the following recursion relations:
\begin{equation}\label{eq:recursion01}
  \mu \phi_0 = (1+z_-)\phi_1 \;,
\end{equation}
\begin{equation}\label{eq:recursion}
  q^y (\phi_{y+1} + \phi_{y-1}) = \mu \phi_y \;, \quad y>0 \;.
\end{equation}
This is difficult to solve. It is useful to note that because of the factor $q^y$, a solution for $\phi_y$ must decrease faster than exponential. For example, something like $\phi_y \sim q^{y^2}$ ($q<1$).

\subsection{Left-Eigenvector Recursion}


By writing the left eigenvector $\bra{\psi}$ as a sum over the basis vectors $\bra{y}$:
\begin{equation}\label{eq:psi_basis}
  \bra{\psi} = \sum_{y=0}^\infty \psi_y \bra{y} \;,
\end{equation}
We find that  coefficients $\psi_y$ satisfy the following recursion relations:
\begin{equation}\label{eq:recursion_psi_1}
  \mu \psi_0 = (1+z_-)\psi_1 \;,
\end{equation}
\begin{equation}\label{eq:recursion_psi_2}
  \mu \psi_1 = (1+z_-)\psi_0 + q^2\psi_2 \;,
\end{equation}
\begin{equation}\label{eq:recursion_psi_3}
   \mu \phi_y = q^{y+1} \phi_{y+1} + q^{y-1} \phi_{y-1} \;, \quad y>1 \;.
\end{equation}
From \eqref{eq:recursion_psi_1} and \eqref{eq:recursion_psi_2} we find the boundary condition
\begin{equation}\label{eq:psi_bc}
  \left( \mu - \frac{c^2}{\mu} \right) \psi_1 = q^2 \psi_2 \;,
\end{equation}
where $c = 1 + z_-$.

\section{Continuum Approximation}

We have found previously (not shown here) that
\begin{equation}\label{eq:z+}
  z_+ = 1 + \OL \;,
\end{equation}
thus
\begin{equation}\label{eq:q}
  q = 1 - \OL = 1 - \epsilon \;, \quad \epsilon \ll 1 \;.
\end{equation}
We can then make a continuum approximation of the recursion relation \eqref{eq:recursion}, to find
\begin{equation}\label{eq:ctm}
  (2-\mu - 2y\epsilon) \phi(y) + \frac{\D^2 \phi}{\D y^2} = 0 + \mbox{higher order terms} \;, %partial derivative?
\end{equation}
and from \eqref{eq:recursion01} we find the boundary condition
\begin{equation}\label{eq:bc}
  \left( \frac{\mu}{1+z_-} -1 \right) \phi(0) = D\phi(0) = \left. \frac{\D \phi}{\D y} \right|_{y=0} \;.
\end{equation}

When $\mu < 2$, the solution to \eqref{eq:ctm} oscillates, which is not what we want, so we disregard this case. 

When $\mu - 2 >> 2y\epsilon$, we find we that we cannot satisfy \eqref{eq:bc}, so we also disregard this case. To see this, observe that $\phi(y)$ becomes the solution to the equation
\begin{equation}
  \frac{\D^2 \phi}{\D y^2} - (\mu-2) \phi(y) = 0 \;,
\end{equation}
with the general solution
\begin{equation}
  \phi(y) = k_1 \e^{y\sqrt{\mu-2}} + k_2 \e^{-y\sqrt{\mu-2}} \;.
\end{equation}
As $y\to\infty$ we require that $\phi(y)$ is finite. Thus, $k_1 = 0$, and
\begin{equation}
  \phi(y) = k_2 \e^{-y\sqrt{\mu-2}} \;.
\end{equation}
Inserting this into the boundary condition \eqref{eq:bc}, we find
\begin{equation}
 -\sqrt{\mu-2} =\left( \frac{\mu}{c} -1 \right) \;,
\end{equation}
where
\begin{equation}
  c = (1+z_-) \;.
\end{equation}
This can be rearranged to give
\begin{equation}
  0 = \mu^2 - (2(1+z_-) + (1+z_-)^2)\mu + (1+z_-)^2 \;,
\end{equation}
and
\begin{equation}
  \mu_\pm = \frac{c}{2} \left( (2+c) \pm \sqrt{ (2+c)^2 - 12}  \right)\;.
\end{equation}
To have real eigenvalues $\mu$, we require
\begin{eqnarray}
  (c+2)^2 - 12 & \ge & 0 \nonumber \\
  (c - c_+)(c-c_-) & \ge & 0 \;,
\end{eqnarray}
where
\begin{equation}
  c_\pm = -2 \pm 2 \sqrt{3} \;.
\end{equation}
Because
\begin{equation}
  0 < c = (1+z_-) < 1 \;,
\end{equation}
we find that
\begin{eqnarray}
  (c-c_+) & < & 0  \nonumber \\
  (c-c_-) & > & 0
\end{eqnarray}
and thus there are no real eigenvalues $\mu$ which satisfy the boundary condition when $\mu - 2 \gg 2y\epsilon$.

This leaves us with the case $\mu = 2 + \delta$, where $\delta$ is a small quantity, like $\epsilon$. In this special case, the equation we want to solve for the eigenvector is
\begin{equation}\label{eq:ctm2}
   \frac{\D^2 \phi}{\D y^2} - (\delta + 2\epsilon y) \phi(y) = 0 \;,
\end{equation}
which is of the form of Airy's equation:
\begin{equation}\label{eq:Airy-eq}
  \frac{\D^2 f(z)}{\D z^2} - zf(z) = 0 \;,
\end{equation}
to which the solution is $f(z) = \Ai(z)$, where 
\begin{equation}\label{eq:Airy-fn}
  \Ai(x) = \frac{1}{\pi} \int_0^\infty \cos\left(\frac{t^3}{3} + xt\right) \D t \;.
\end{equation}
So to find $\phi(y)$ we make a change of variables:
\begin{equation}
  \phi(y) \to f(z) \;, \quad z = A + By \;.
\end{equation}
This gives us
\begin{equation}
  B^2 \frac{\D^2 f}{\D z^2} - \left( \delta + 2\epsilon \frac{(z-A)}{B} \right) f(y) = 0 \;,
\end{equation}
To bring this into the form of Airy's equation \eqref{eq:Airy-eq}, we require
\begin{equation}
  \delta - \frac{2\epsilon A}{B} = 0 \;, \quad \frac{2\epsilon}{B^3} = 1 \;,
\end{equation}
which tells us that 
\begin{equation}
  B = (2\epsilon)^{1/3} \;, \quad \delta = (2\epsilon)^{2/3} A \;.
\end{equation}
So we find that 
\begin{equation}\label{eq:phi_gen_soln}
  \phi(y) = \alpha \Ai(A + (2\epsilon)^{1/3}y) \;,
\end{equation}
where $\alpha$ is some constant. We can get an approximate value for $A$ by considering the boundary condition \eqref{eq:bc}. Because $0 < 1+z_- < 1$, $D > 0$. This means that
\begin{equation}
 \frac{\phi'(0)}{\phi(0)} = D > 0 \;, \quad \phi'(0) = \left. \frac{\D \phi(y)}{\D y} \right|_{y=0} \;,
\end{equation}
and so
\begin{equation}
  \frac{\Ai'(A)}{\Ai(A)} = \frac{D}{(2\epsilon)^{1/3}} >> 1 \;, \quad \Ai'(A) = \left. \frac{\D \Ai(z)}{\D y} \right|_{y=0} \;.
\end{equation}
We can see from the plot in Figure \ref{fig:Airy-fn} that this condition can be satisfied near $x=z_0$, the first real root of the Airy function.
\begin{figure}[h!]
  \centering
  \includegraphics[width=0.5\textwidth]{airy}
  \caption{Plot of the Airy function, $\Ai(x)$, as given in \eqref{eq:Airy-fn}}
  \label{fig:Airy-fn}
\end{figure}
$\phi(y)$ must be positive and not infinite, however, so
\begin{equation}
  A = z_0 + \Or( (2\epsilon)^{1/3} )  = z_0 + \beta (2\epsilon)^{1/3} \;,
\end{equation}
where $\beta$ is some finite constant. So finally, we can write
\begin{equation}
  \phi(y) = \alpha \Ai \bigg( (y+\beta)(2\epsilon)^{1/3} - |z_0| \bigg)
\end{equation}


\subsection{Continuum Limit of the Left-Eigenvector}

Again, using \eqref{eq:z+} and \eqref{eq:q} we can make a continuum approximation for $\psi(y)$:
\begin{equation}
  (2-\mu - 2y\epsilon) \psi(y) + 2\epsilon \frac{\D \psi}{\D y} + \frac{\D^2 \psi}{\D y^2} = 0 + \mbox{higher order terms} \;.
\end{equation}
Assuming the second term
\begin{equation*}
  2\epsilon \frac{\D \psi}{\D y}
\end{equation*}
can be neglected with the other higher order terms {\bf (NOTE: 2015-06-15: Can it be negledted with the higher order terms???)}, then we recover the same equation \eqref{eq:ctm} as we did for $\phi$: 
\begin{equation}\label{eq:ctm_psi}
  (2-\mu - 2y\epsilon) \psi(y) + \frac{\D^2 \psi}{\D y^2} = 0 + \mbox{higher order terms} \;.
\end{equation}
However, from \eqref{eq:psi_bc}, we find a different boundary condition:
\begin{equation}
   \tilde{D}\psi(1) = \left(\mu - \frac{c^2}{\mu} - 1 \right) \psi(1) = \left. \frac{ \D \psi }{\D y} \right|_{y=1} \;.
\end{equation}
Thus, $\psi(y)$ solves the same differential equation as $\phi(y)$, except with a different boundary condition:
\begin{equation}\label{eq:psi_gen_soln}
  \psi(y) = \alpha \Ai(A + (2\epsilon)^{1/3}y) \;.
\end{equation}
As with $\phi(y)$,
\begin{equation}
 \frac{\psi'(0)}{\psi(0)} = \tilde{D} > 0 \;,
\end{equation}
and so
\begin{equation}
  \frac{\Ai'(A)}{\Ai(A)} = \frac{\tilde{D}}{(2\epsilon)^{1/3}} >> 1 \;.
\end{equation}
$A$ is obtained using the boundary condition again, but with a slight modification due the the boundary condition being at $y=1$, not $0$:
\begin{equation}
  A +(2\epsilon)^{1/3} = z_0 + \Or( (2\epsilon)^{1/3} )  = z_0 + \beta (2\epsilon)^{1/3} \;,
\end{equation}
and so
\begin{equation}
  \psi(y) = \alpha \Ai(z_0 + (y+\beta-1)(2\epsilon)^{1/3}) \;.
\end{equation}


\subsection{Computing Properties of $P(y)$}

From \eqref{eq:phi_basis}, 
\begin{equation}
  \ket{\phi} = \int_{0}^\infty  \D y\; \phi(y') \ket{y'}\;,
\end{equation}
thus
\begin{equation}
  \braket{y}{\phi} = \phi(y) \;.
\end{equation}
Similarly,
\begin{equation}
  \braket{\psi}{y} = \psi(y)\;.
\end{equation}
Now we can write
\begin{equation}
  P(0) = \frac{c}{\mu} \frac{\psi(1)\phi(0)}{Z} \;,
\end{equation}
and
\begin{equation}
 P(y) = \frac{ \psi(y)\phi(y)}{Z} \;, \quad y \ge 1 \;,
\end{equation}
where
\begin{equation}
  Z = \frac{c}{\mu}\psi(1)\phi(0)  + \int_{1}^\infty \D y' \; \psi(y')\phi(y') \;.
\end{equation}
To calculate the $n$-th moment of $P(y)$, we need to compute
\begin{equation}
  \langle y^n \rangle = \frac{1}{Z} \left( \frac{c}{\mu}\psi(1)\phi(0)  + \int_{1}^\infty \D y \; \psi(y)\phi(y) \right) \;.
\end{equation}
{\bf (NOTE: I'm not convinced. It seems I have an expression for $P(y)$ for $y=0$ and $y\in[1,\infty)$, but not for $y\in(0,1)$??)}

%  OLD STUFF:
% \begin{equation}
%   \langle y^n \rangle = \int_0^\infty \D y \; \frac{ y^n \phi(y)^2 }{Z} \;,
% \end{equation}
% which means computing
% \begin{equation}
%   \langle y^n \rangle = \int_0^\infty \D y \; \frac{\alpha^2 y^n }{Z} \left[ \Ai \bigg( (y+\beta)(2\epsilon)^{1/3} - |z_0| \bigg) \right]^2 \;,
% \end{equation}
% where
% \begin{equation}
%  Z = \int_0^\infty \D y' \; \alpha^2 \left[ \Ai \bigg( (y'+\beta)(2\epsilon)^{1/3} - |z_0| \bigg) \right]^2\;.
% \end{equation}
% The factors of $\alpha^2$ cancel, so 
% \begin{equation}
% %   \langle y^n \rangle = \frac{ \int_0^\infty \D y \; y^n \left[ \Ai \bigg( (y+\beta)(2\epsilon)^{1/3} - |z_0| \bigg) \right]^2 }{ \int_0^\infty \D y' \; \left[ \Ai \bigg( (y'+\beta)(2\epsilon)^{1/3} - |z_0| \bigg) \right]^2 } \;.
% \end{equation}

\subsection{Scaling}

The main thing that we want to get from this analysis is how properties such as the density of contacts, $\rho$, the mean distance from the membrane, $\langle y \rangle$, and the width $W = \sqrt( \langle y^2 \rangle - \langle y \rangle^2)$ scale with $L$. 
% From previous analysis
% \begin{equation}
%   z_+^{-1} = q = 1 - \epsilon = 1 - \frac{1}{4u+1} \frac{d}{L} + \OLL \;.
% \end{equation}
% Thus
% \begin{equation}
%   \epsilon \sim \frac{1}{L} \;,
% \end{equation}
% and so all we really need to find is how $\langle y^n \rangle$ scales with $\epsilon$. {\bf (NOTE: actually, see below: have to find how $P(0)$ scales with $\epsilon$ before finding how $\epsilon$ scales with $P(0)$.)}
We write
\begin{equation}
  z_+^{-1} = q = 1 - \epsilon \;, \quad \epsilon \ll 1 \;, 
\end{equation}


% Using the substitution
% \begin{equation}
%   z = (y+\beta)(2\epsilon)^{1/3} + z_0 \;,
% \end{equation}
% \begin{equation}
%   \frac{\D y}{ \D z} = (2\epsilon)^{-1/3} \;,
% \end{equation}
% and the limits of integration are
% \begin{eqnarray}
%   y=0 \;: & & \quad z = \beta(2\epsilon)^{1/3} + z_0 \nonumber \\
%   y\to\infty \;: & & \quad z\to\infty 
% \end{eqnarray}
% Also, 
% \begin{equation}
%   y = (2\epsilon)^{-1/3}(z-z_0) - \beta = (2\epsilon)^{-1/3}(z-z_0) \left( 1 - \frac{\beta}{(2\epsilon)^{-1/3}(z-z_0)} \right) \;.
% \end{equation}
% Because $\epsilon^{-1/3} \gg 1$,
% \begin{equation}
%   \frac{\beta}{(2\epsilon)^{-1/3}(z-z_0)} \ll 1 \;,
% \end{equation}
% and so
% \begin{equation}
%   y^n \simeq (2\epsilon)^{-n/3}(z-z_0)^n \left( 1 - \frac{n\beta}{(2\epsilon)^{-1/3}(z-z_0)} \right) \;.
% \end{equation}
% 
% We can now substitute all of these things into the integral above:
% % \begin{eqnarray}
%   & \displaystyle\int_0^\infty \D y \; y^n \left[ \Ai \big( (y+\beta)(2\epsilon)^{1/3} - |z_0| \big) \right]^2 & = \cdots  \nonumber \\
%   \cdots = & (2\epsilon)^{-(n+1)/3} \displaystyle\int_{\beta(2\epsilon)^{1/3}+z_0} ^\infty \D z \; (z-z_0)^n \big[ \Ai(z) \big]^2 & \;,
% \end{eqnarray}
% so
% \begin{equation}
%   \langle y^n \rangle = (2\epsilon)^{-n/3} \frac{ \displaystyle\int_{\beta(2\epsilon)^{1/3}+z_0} ^\infty \D z \; (z-z_0)^n \big[ \Ai(z) \big]^2 }{ \displaystyle\int_{\beta(2\epsilon)^{1/3}+z_0} ^\infty \D z \; \big[ \Ai(z) \big]^2 } \;.
% \end{equation}
% Unfortunately, the two integrals do not cancel because of the factor $(z+z_0)^n$. A plot of the region over which the integral is performed is shown in Figure \ref{fig:xnAiry2}, for the alternative substitution
%  \begin{equation}
%    x = (y+\beta)(2\epsilon)^{1/3} \;.
%  \end{equation}
\begin{figure}[h!]
  \centering
  \includegraphics[width=0.5\textwidth]{xn_plusz0_airy2}
  \caption{Plot of the (significant part of the) region of integration over $x^n \Ai(x+z_0)^2$, for $n = 0,1,2$. $z_0 \simeq -2.338110$ (\emph{WolframAlpha}).}
  \label{fig:xnAiry2}
\end{figure}
% If $\beta=0$ (or $\beta(2\epsilon)^{1/3}$ is small enough relative to $z_0$ that we can discard it) then, even though the integrals do not cancel, they no longer have any $\epsilon$ dependance, so in this case
% \begin{equation}
%   \langle y^n \rangle \sim \epsilon^{-n/3} \;.
% \end{equation}

To determine how $\epsilon$ is related to $L$ we must use the fact that $\pzero \sim L^{-1}$ and find how $\pzero$ scales with $\epsilon$.
\begin{equation}
  \pzero = \frac{\psi(1)\phi(0)}{Z}
\end{equation}
and
\begin{equation}
  \psi(1)\phi(0) = \alpha^2 [ \Ai(z_0 + \beta(2\epsilon)^{1/3}) ]^2 \;.
\end{equation}
Using
\begin{eqnarray}
  \Ai(z_0 + \delta) & \simeq & \Ai'(z_0)\delta + \frac{\Ai''(z_0)}{2} \delta^2 + \ldots \nonumber \\
  \bigg( \Ai(z_0 + \delta) \bigg)^2 & \simeq & \bigg( \Ai'(z_0) \bigg)^2 \delta^2 + \Or(\delta^3) \;,
\end{eqnarray}
with
\begin{equation}
  \delta = \beta^2(2\epsilon)^{2/3} \;,
\end{equation}
we see that
\begin{equation}
  \psi(1)\phi(0) \sim \epsilon^{2/3} \;.
\end{equation}
$Z$ also scales with epsilon. 
\begin{equation}
 Z = \frac{c}{\mu}\psi(1)\phi(0)  + \int_{1}^\infty \D y \; \psi(y)\phi(y)  = \frac{c}{\mu}\psi(1)\phi(0) + \tilde{Z}\;.
\end{equation}
Explicitly, $\tilde{Z}$ is
\begin{equation}
 \tilde{Z} = \int_{1}^\infty \D y \; \alpha^2 \Ai( z_0 + (y+\beta-1)(2\epsilon)^{1/3}) \Ai ( z_0 + (y+\beta)(2\epsilon)^{1/3}) \;.
\end{equation}
Using the susbstitution
\begin{equation}
  x = (y+\beta-1)(2\epsilon)^{1/3} \;,
\end{equation}
\begin{equation}
 \tilde{Z} = \alpha^2 (2\epsilon)^{-1/3} \int_{\beta(2\epsilon)^{1/3}}^\infty \D y \;  \Ai( z_0 + x) \Ai ( z_0 + x + (2\epsilon)^{1/3}) \;.
\end{equation}
We then expand in terms of $(2\epsilon)^{1/3}$, to get 
\begin{eqnarray}
 \tilde{Z}  & = & (2\epsilon)^{-1/3} \alpha^2  \int_{\beta(2\epsilon)^{1/3}}^\infty \D y \;  \Ai( z_0 + x) \Ai ( z_0 + x)  \nonumber \\
            & + & \alpha^2  \int_{\beta(2\epsilon)^{1/3}}^\infty \D y \;  \Ai( z_0 + x) \Ai' ( z_0 + x )  \nonumber \\
            & + & \Or(2\epsilon)^{1/3}\;.
\end{eqnarray}
% Using the substitution above for $y$ in terms of $x$ given above, 
% \begin{equation}
%   Z = (2\epsilon)^{-1/3} \int_{\beta(2\epsilon)^{1/3}}^\infty [ \Ai ( x + z_0 ) ]^2 \D x \;.
% \end{equation}
% Thus,
% \begin{equation}
%   Z \sim \epsilon^{-1/3}
% \end{equation}
% and
% \begin{equation}
%   P(0) \sim \frac{ \epsilon^{2/3} }{ \epsilon^{-1/3} } \sim \epsilon^1 \;.
% \end{equation}
% This means that
% \begin{equation}
%   \epsilon \sim L^{-1}
% \end{equation}
% and
% \begin{equation}
%   \langle y^n \rangle \sim L^{n/3} \;,
% \end{equation}
% which gives
% \begin{equation}
%   \langle y \rangle \sim L^{1/3} \;, \quad W \sim L^{1/3} \;.
% \end{equation}
So
\begin{equation}
  Z \sim \epsilon^{2/3} + \epsilon^{-1/3} \;,
\end{equation}
and
\begin{equation}
  P(0) \sim \frac{\epsilon^{2/3}}{\epsilon^{2/3} + \epsilon^{-1/3}} = \frac{\epsilon}{\epsilon + 1} \sim \epsilon \;.
\end{equation}
This means that
\begin{equation}
  \epsilon \sim \frac{1}{L} \;,
\end{equation}
and so
\begin{equation}
  \langle y \rangle \sim L^{1/3} \;, \quad W \sim L^{1/3} \;.
\end{equation}
{\bf (NOTE: this comes from $\langle y^n \rangle$ scaling, which I've assumed is unchanged...)}

\section{Reduced Transfer Matrix}

\emph{(See Section \ref{sec:K_thoughts} for my thoughts on how this might tell us about how the width, etc... scales with $L$.)}\newline

The quantity
\begin{equation}
  P(y) = \frac{ \bra{y} T^L \ket{y} }{ \sum_{y'} \bra{y'} T^L \ket{y'} }
\end{equation}
is the probability that a single site picked at random has height $y$. In the mean field equation we used the probability $K$ that the interface \emph{does not} touch the membrane. Explicitly, we chose
\begin{equation}\label{eq:K_no_correl}
  K = (1-P(0))^L \;.
\end{equation}
This choice assumes that there are no correlations between the neighbouring sites on the interfce, however now that we've started using the transfer matrix $T$, we have started to put nearest-neightbour height restrictions back in. It follows then that the choice in \eqref{eq:K_no_correl} may not be the best one.

We can define a reduced transfer matrix $\tilde{T}$ which selects only configurations where $\emph{none}$ of the heights are $0$. There are two possible definitions ot $\tilde{T}$ that can do this:
\begin{equation}
  \tilde{T}_r = T(\mathbb{1}-\ket{0}\bra{0}) = T\sum_{y=1}^\infty \ket{y}\bra{y} \;,
\end{equation}
\begin{equation}
  \tilde{T}_r = \begin{pmatrix}
            0 & c   & 0   & 0   & 0   & \cdots \\
            0 & 0   & q   & 0   & 0   &        \\
            0 & q^2 & 0   & q^2 & 0   &        \\
            0 & 0   & q^3 & 0   & q^3 &        \\
       \vdots &     &     &     &     & \ddots \\
      \end{pmatrix} \;;
\end{equation}
or
\begin{equation}
  \tilde{T}_l = (\mathbb{1}-\ket{0}\bra{0})T = \sum_{y=1}^\infty \ket{y}\bra{y}T \;,
\end{equation}
\begin{equation}
  \tilde{T}_r = \begin{pmatrix}
            0 & 0   & 0   & 0   & 0   & \cdots \\
            q & 0   & q   & 0   & 0   &        \\
            0 & q^2 & 0   & q^2 & 0   &        \\
            0 & 0   & q^3 & 0   & q^3 &        \\
       \vdots &     &     &     &     & \ddots \\
      \end{pmatrix} \;.
\end{equation}

Whichever definition we use, the probability have a configuration does not touch the membrane is
\begin{equation}
  K = \frac{ \sum_y \bra{y} \tilde{T}^L \ket{y} }{\sum_{y'} \bra{y'} T^L \ket{y'}}  = \frac{1}{Z} \sum_y \bra{y} \tilde{T}^L \ket{y} \;,
\end{equation}
where $Z$ is the same partition function for all configurations given in the previous sections. To calculate $K$ we need to find the eigenvectors of $\tilde{T}$ in the same way we did for $T$. 

\subsection{Eigevectors of $\tilde{T}_r$}

The right eigenvectors $\phi_{r,y}$ satisfy
\begin{eqnarray}
  c\phi_{r,1} & = & \mu \phi_{r,0} \;, \\
  q \phi_{r,2} & = & \mu \phi_{r,1} \;, 
\end{eqnarray}
and
\begin{equation}
  q^y (\phi_{r,y-1} + \phi_{r,y+1} ) = \mu \phi_{r,y} \;, \quad y > 1 \;.
\end{equation}

The left eigenvectors of of $\tilde{T}_r$ satisfy
\begin{eqnarray}
  0 & = & \mu \psi_{r,0} \;, \\
  c \psi_{r,0} + q^2 \psi_{r,2} & = & \mu \psi_{r,1} \;,
\end{eqnarray}
and
\begin{equation}
  q^{y-1} \psi_{r,y-1} + q^{y+1} \psi_{r,y+1} = \mu \psi_{r,y} \;, \quad y > 1 \;.
\end{equation}


\subsection{Eigevectors of $\tilde{T}_l$}

The right eigenvectors $\phi_{l,y}$ satisfy
\begin{eqnarray}
  0 & = & \mu \phi_{l,0} \;, \\
  q (\phi_{l,0} + \phi_{l,2}) & = & \mu \phi_{l,1} \;, 
\end{eqnarray}
and
\begin{equation}
  q^y (\phi_{l,y-1} + \phi_{l,y+1} ) = \mu \phi_{l,y} \;, \quad y > 1 \;.
\end{equation}

The left eigenvectors of of $\tilde{T}_l$ satisfy
\begin{eqnarray}
  q \psi_{l,1} & = & \mu \psi_{l,0} \;, \\
  q^2 \psi_{l,2} & = & \mu \psi_{l,1} \;,
\end{eqnarray}
and
\begin{equation}
  q^{y-1} \psi_{l,y-1} + q^{y+1} \psi_{l,y+1} = \mu \psi_{l,y} \;, \quad y > 1 \;.
\end{equation}

\subsection{Continuum Limit}

We can take the continuum limit in the same way as before:
\begin{equation}
  q = 1 - \epsilon \;,
\end{equation}
\begin{equation}
  \chi_{y\pm 1} = \chi(y) \pm \frac{\D \chi }{\D y} + \frac{1}{2} \frac{ \D^2 \chi}{\D y^2} + \mbox{h.o.t.} \;,
\end{equation}
where for notation's sake $\chi(y)$ can be either $\phi(y)$ or $\psi(y)$. For both $\phi$ and $\psi$, and both $\tilde{T}_r$ and $\tilde{T}_l$, the eigenfunctions satisfy the same differential equations as the eigenfunctions of $T$:
\begin{equation}\label{eq:ctm}
  (2-\mu - 2y\epsilon) \phi(y) + \frac{\D^2 \phi}{\D y^2} = 0 + \mbox{higher order terms} \;, %partial derivative?
\end{equation}
\begin{equation}
  (2-\mu - 2y\epsilon) \psi(y) + 2\epsilon \frac{\D \psi}{\D y} + \frac{\D^2 \psi}{\D y^2} = 0 + \mbox{higher order terms} \;.
\end{equation}
The difference comes in the boundary conditions for $\phi_r(y)$, $\psi_r(y)$, $\phi_l(y)$ and $\psi_l(y)$.

\subsubsection{Boundary conditions for $\tilde{T}_r$}

The relationship
\begin{equation}
  (1-\epsilon) \phi_{r,2} = \mu \phi_{r,1}
\end{equation}
gives the boundary condition
\begin{equation}\label{eq:rr_bc}
  \left. \frac{ \D \phi_r}{\D y} \right|_{y=1} = [\mu - 1 + \mu \epsilon + \Or(\epsilon^2) ] \phi_r(1) \;, \quad \mu \ne 0 \;.
\end{equation}
 
The relationship
\begin{equation}
 (1-\epsilon^2) \psi_{r,2} = \psi_{r,1}   
\end{equation}
gives the boundary condition
\begin{equation}\label{eq:rl_bc}
  \left. \frac{ \D \psi_r}{ \D y} \right|_{y=1} = [\mu -1  + 2\mu\epsilon + \Or(\epsilon^2) ] \psi_r(y) \;, \quad \mu \ne 0 \;.
\end{equation}

\subsubsection{Boundary conditions for $\tilde{T}_l$}

The relationship
\begin{equation}
  (1-\epsilon)\phi_{l,2} = \mu \phi_{l,1}
\end{equation}
gives the boundary condition
\begin{equation}\label{eq:lr_bc}
  \left. \frac{ \D \phi_l}{\D y} \right|_{y=1} = [\mu - 1 + \mu \epsilon + \Or(\epsilon^2) ] \phi_l(1) \;, \quad \mu \ne 0 \;.
\end{equation}

The relationship
\begin{equation}
  (1-\epsilon)^2 \psi_{l,2} = \mu \psi_{l,1} 
\end{equation}
gives the boundary condition
\begin{equation}\label{eq:ll_bc}
  \left. \frac{ \D \psi_l}{ \D y} \right|_{y=1} = [\mu -1  + 2\mu\epsilon + \Or(\epsilon^2) ] \psi_l(y) \;, \quad \mu \ne 0 \;.
\end{equation}

\subsubsection{Differences at $y=0$}

So we see that the only difference between the continuum limit eigenfunctions of $\tilde{T}_r$ and $\tilde{T}_l$ come in at $y=0$:
\begin{eqnarray}
  c\phi_{r,1} = \mu \phi_{r,0} & \to & \phi_r(0) = \frac{c}{\mu} \phi_r(1) \\
  0 = \mu \psi_{r,0} & \to & \psi_r(0) = 0 \;, \mu \ne 0 \\
  0 = \mu \phi_{l,0} & \to & \phi_l(0) = 0 \;, \mu \ne 0 \\
  q \psi_{l,1} = \mu \psi_{l,0} & \to & \psi_l(0) = \frac{(1-\epsilon)}{\mu} \;.
\end{eqnarray}
 This is good because it means that
\begin{equation}
  P(0) = \frac{\psi_x(0)\phi_x(0)}{Z} = 0 \;,
\end{equation}
whichever definition of $\tilde{T}$ you use ($x = l,r$).

\subsubsection{Solution}

So now we can do away with the subscript $r$,$l$ notation. The solution for $\phi(y)$ is the same as the earlier result \eqref{eq:phi_gen_soln}:
\begin{equation}
 \phi(y) = \alpha \Ai(A + (2\epsilon)^{1/3} y) \;,
\end{equation}
and likewise, from \eqref{eq:psi_gen_soln},
\begin{equation}
  \psi(y) = \alpha \Ai(A + (2\epsilon)^{1/3} y)
\end{equation}
as well.

Now we look at the boundary conditions. From \eqref{eq:rr_bc} and \eqref{eq:lr_bc}, the boundary condition(s) for $\phi(y)$, we find
\begin{eqnarray}
  (2\epsilon)^{1/3} \frac{\Ai'( A + (2\epsilon)^{1/3} )}{\Ai(A + (2\epsilon)^{1/3})} & = & \mu -1 + \mu \epsilon + \Or(\epsilon^2) \;, \\
                    \frac{\Ai'( A + (2\epsilon)^{1/3} )}{\Ai(A + (2\epsilon)^{1/3})} & = & \frac{\mu -1}{(2\epsilon)^{1/3}} + \Or(\epsilon^{2/3}) \;,
\end{eqnarray}
where again
\begin{equation}
  \Ai'(z) = \left. \frac{ \D \Ai(\tilde{z})}{\D \tilde{z}} \right|_{\tilde{z}=z} \;.
\end{equation}
Just as before, as $\epsilon\to0$, this boundary condition is satisfied when
\begin{equation}
  A + (2\epsilon)^{1/3} = z_0 + \beta(2\epsilon)^{1/3} + \Or((2\epsilon)^{2/3}) \;.
\end{equation}
Thus,
\begin{equation}
  \phi(y) = \alpha \Ai( z_0 + (y + \beta - 1)(2\epsilon)^{1/3} ) \;.
\end{equation}
For $\psi(y)$, using the boundary condition(s) \eqref{eq:rl_bc} and \eqref{eq:ll_bc}, an almost identical analysis follows, yielding an identical solution:
\begin{equation}
  \psi(y) = \alpha \Ai( z_0 + (y + \beta - 1)(2\epsilon)^{1/3} ) \;.
\end{equation}
The constants $\alpha$ and $\beta$ are not necessarily the same for both however, so we label them $\alpha_\phi$, $\beta_\phi$, $\alpha_\psi$, and $\beta_\psi$ from now on.
So finally, we have found
\begin{equation}
  \bra{y}\tilde{T}^L\ket{y} = \alpha_\phi\alpha_\psi\int_1^\infty \D y \; \Ai[ z_0 + (y + \beta_\phi - 1)(2\epsilon)^{1/3} ]\Ai[ z_0 + (y + \beta_\psi - 1)(2\epsilon)^{1/3} ] \;.
\end{equation}

\subsubsection{Scaling}

We can perform the same scaling analysis as previously. First, use the substitution
\begin{equation}
 x = (y + \beta_\phi -1)(2\epsilon)^{1/3} \;,
\end{equation}
to write
\begin{equation}
  \bra{y}\tilde{T}^L\ket{y} = \alpha_\phi\alpha_\psi (2\epsilon)^{-1/3} \int_{\beta_\phi(2\epsilon)^{1/3}}^\infty \D x \; \Ai[ z_0 + x ]\Ai[ z_0 + x (\beta_\psi - \beta_\phi)(2\epsilon)^{1/3} ] \;.
\end{equation}
Assuming that the integral scales like $\epsilon^0$ to leading order in $\epsilon$, then
\begin{equation}
  \bra{y}\tilde{T}^L\ket{y} \sim (2\epsilon)^{-1/3} \sim \epsilon^{-1/3} \;.
\end{equation}
From previous calculation, $Z$ scales like $\epsilon^{-1/3}$, so
\begin{equation}
  K \sim \frac{ \epsilon^{-1/3} }{ \epsilon^{-1/3} } \sim \epsilon^0 \;.
\end{equation}


\subsection{Understanding the scaling - the relationship with $L$?}\label{sec:K_thoughts}

When we defined $K = (1-P(0))^L$ it was clear how the system size $L$ came into the equations via $P(0)$, and we could link $\epsilon$ to $L$, and thus $\overline{y}$ and $W^2$ to $L$. Now, with a general $K$, which we try to calculate using the transfer matrix $\tilde{T}$, $L$ doesn't come into it anywhere! In fact, it's not even clear that $q \simeq 1- \epsilon$. It is only clear that $z_+ = z_+(u,K)$ and $z_- = z_-(u,K)$. What does this mean? Well, probably that the Airy function analysis, and the continuum limit using $\epsilon$, are not valid. $T = T(u,K)$ and $\tilde{T} = \tilde{T}(u,K)$, so maybe we have to find $K$ in a horrible self-consistent way. That still doesn't tell us how to get $L$ involved. Perhaps we have then to guess $P(0) \sim L^{-1}$ as our simulations suggest, and then use this and $K$ to find how the width scales with $L$?




\section{Transfer Matrix with $q=1$}

We can see what scaling we get when $q=1$. As can be seen from Figure \ref{fig:evects-epsilon}, when $\epsilon > 0$ the eigenvectors decay fast, as found above, and when $\epsilon =0$, the eigenvectors are extended and periodic, and there is a continuous spectrum of eigenvalues.
\begin{figure}
  \centering
  \includegraphics[width=\textwidth]{eigs_N500c05_e-range.pdf}
  \label{fig:evects-epsilon}
\end{figure}

We proceed by redefining the transfer matrix $T$ with $q=1$:
\begin{equation}
  T = \begin{pmatrix}
       0 & c & 0 & 0 & 0 & \cdots \\
       1 & 0 & 1 & 0 & 0 &        \\
       0 & 1 & 0 & 1 & 0 &        \\
       0 & 0 & 1 & 0 & 1 &        \\
       \vdots &  &  &  &  & \ddots \\
      \end{pmatrix} \;,
\end{equation}
where $c=(1+z_-) <1$. Solving for the right eigenvectors 
\begin{equation}
  \ket{\phi(\mu)} = \sum_{y=0}^\infty \phi_y(\mu) \ket{y}
\end{equation}
and eigenvalues $\mu$ gives us the recursion relation
\begin{equation}
  \mu\phi_y = \phi_{y+1} + \phi_{y-1} \;, \quad y>0 \;,
\end{equation}
with the boundary condition
\begin{equation}
  \mu\phi_0 = c\phi_1 \;.
\end{equation}
For the left eigenvectors
\begin{equation}
  \bra{\psi(\mu)} = \sum_{y=0}^\infty \psi_y(\mu) \bra{y} \;,
\end{equation}
with the same eigenvalues $\mu$, we find the same recursion relation
\begin{equation}
  \mu \psi_y = \psi_{y+1} + \psi_{y-1} \;,
\end{equation}
but with different boundary conditions:
\begin{equation}
  \mu \psi_0 = \psi_1 
\end{equation}
and
\begin{equation}
  \mu \psi_1 = c \psi_0 + \psi_2 \;.
\end{equation}


\subsection{Trial Solution}
We proceed with the trial solution
\begin{equation}
  \phi_y = At^y + Bt^{-y} \;.
\end{equation}
With simple substitution it can be shown that this solution satisfies the recursion with the condition
\begin{equation}
  \mu = t_+ + t_+^{-1} = t_- + t_-^{-1} \;,
\end{equation}
Where
\begin{equation}
  t_\pm = \frac{\mu \pm (\mu^2-4)^{1/2}}{2} \;
\end{equation}
and
\begin{equation}
  t_+ = t_-^{-1} \;.
\end{equation}
Thus
\begin{equation}
  \phi_y = At_+^y + Bt_-^y \;.
\end{equation}
It is useful at this point to note some convenient relationships between $t$ and $\mu$:
\begin{equation}
  t_+ + t_- = \mu \;,
\end{equation}
\begin{equation}
  t_+ - t_- = (\mu^2 -4)^{1/2} \;,
\end{equation}
\begin{equation}
  t_+t_- = 1 \;,
\end{equation}
and
\begin{equation}
  t_+^2 + t_-^2 = \mu^2 - 2 \;.
\end{equation}
We can also try the same solution for $\psi$:
\begin{equation}
  \psi_y = \tilde{A}t_+^y + \tilde{B}t_-^y \;.
\end{equation}
There are three different solutions depending on $\mu$. 

\subsubsection{$\mu > 2$}

When $\mu > 2$, both $t_+$ and $t_-$ are real and positive. Thus,
\begin{equation}
  t_+ > t_- \;
\end{equation}
and
\begin{equation}
  t_+ > 1 \;, \quad t_- < 1 \;.
\end{equation}
As $y\to \infty$, $\phi_y \to At_+^y$, which is unbounded because $t_+>0$. Thus, $A=0$ and
\begin{equation}
  \phi_y = Bt_-^y \;.
\end{equation}
If we fit this to the boundary condition, we find
\begin{equation}
  t_- = \frac{\mu}{c} \;.
\end{equation}
If we substitute this back in to the equation for $t_-$ we find
\begin{equation}
  \frac{c-2}{2c} = \frac{\mu^2-4}{\mu^2} \;.
\end{equation}
Thus, for this solution to exist we require $c > 2$, but in our system $c < 1$, so we cannot use solutions with $\mu > 2$.


\subsubsection{$\mu =2$}
When $\mu = 2$,
\begin{equation}
  t_+ = t_- = \frac{\mu}{2} = 1 \;.
\end{equation}
So, in this case, 
\begin{equation}
  \phi_y = A + B
\end{equation}
for all $y$. Furthermore, the boundary condition can only be satisfied if
\begin{equation}
  A = -B \;, \quad \mbox{or} \quad A = B = 0 \;.
\end{equation}
Either way, this gives $\phi_y=0$ which is not a useful solution.



\subsubsection{$\mu < 2$}
When $\mu < 2$, $t_\pm$ are complex:
\begin{equation}
  t_\pm = t\mathrm{e}^{\pm i\theta} \;,
\end{equation}
\begin{equation}
  t_+t_- = t^2 = 1 \;.
\end{equation}
Also,
\begin{equation}
  t_\pm = t_\mp^* \;, \quad t_\pm^* = t_\pm^{-1} \;.
\end{equation}
Thus, 
\begin{equation}
  \mu = 2\cos(\theta)\;,
\end{equation}
with
\begin{eqnarray}
  0 < & \theta & < \pi \nonumber \\
  -2 < & \mu & < 2 
\end{eqnarray}
The boundary condition for the right egenvectors $\mu\phi_0=c\phi_1$ gives
\begin{eqnarray}
  \frac{\mu}{c}(A+B) & =&  A t_+ + Bt_- \nonumber \\
  \left[ \mu \frac{1-c}{c} + t_- \right] A & = & -\left[ \mu\frac{1-c}{c} + t_+ \right] B \nonumber \\
  \left[ \frac{\mu}{c} - \mu + t_- \right] A & = & -\left[ \frac{\mu}{c} - \mu + t_+ \right] B \nonumber \\
  \left[ \frac{\mu}{c} - t_+ \right] A & = & -\left[ \frac{\mu}{c} - t_- \right] B \nonumber \\
  \frac{B}{A} & = & -\frac{(\mu - ct_+)}{ (\mu - ct_-)}  \;,
\end{eqnarray}
Thus
\begin{equation}
  \phi_y = A \left[ t_+^y - \frac{(ct_+ - \mu)}{ (ct_- - \mu)} t_-^y \right] \;,
\end{equation}
for all $y\ge0$.

For the left eigenvector, we substitute the first boundary condition $\mu\psi_0 = \psi_1$ into second boundary condition $\mu\psi_1 = c\psi_0+\psi_2$ to get
\begin{equation}
  \frac{\mu^2-c}{\mu} \psi_1 = \psi_2 \;.
\end{equation}
Using the solution $\psi_y = \tilde{A} t_+^y + \tilde{B} t_-^y$ we find
\begin{equation}
  \frac{\tilde{B}}{\tilde{A}} = - \frac{t_+}{t_-} \frac{(t_+\mu-\mu^2+c)}{(t_-\mu-\mu^2+c)}
\end{equation}
which can be simplified to 
\begin{equation}
  \frac{\tilde{B}}{\tilde{A}} = - \frac{(c t_+ - \mu)}{(ct_- - \mu )}\;.
\end{equation}
So we find the components of left eigenvector $\psi_y$ are
\begin{equation}
  \psi_y = \tilde{A} \left[ t_+^y - \frac{(c t_+ - \mu)}{(ct_- - \mu )} t_-^y \right] \;.
\end{equation}
for $y>0$. For $y=0$, we use the first boundary condition to find
\begin{equation}
  \psi_0 = \frac{\tilde{A}}{\mu} \left[ t_+ - \frac{(ct_+-\mu)}{(ct_--\mu)} t_- \right] \;.
\end{equation}
Note that, for $y>0$, $\psi_y = \phi_y$. $\psi_0 \ne \phi_0$.

\subsubsection{Some Useful Related Algebra}

We have $-2< \mu < 2$, with 
\begin{equation}
  \phi_y = A \left[ t_+^y - r t_-^y \right] \;,
\end{equation}
where
\begin{equation}
  r = \frac{(ct_+ - \mu)}{ (ct_- - \mu)} \;.
\end{equation}

Note that $r^*$ = $r^{-1}$, so we can write
\begin{equation}
  r = \mathrm{e}^{i\rho(\theta)} \;.
\end{equation}
With a bit of algebra, one can show that
\begin{equation}
  r(\mu(\theta)) = \frac{ (c-1)^2\mu^2 - (4-\mu^2) + 2i(c-1)\mu(4-\mu^2)^{1/2} }{ (c-1)^2\mu^2 + 4 - \mu^2 } \;, 
\end{equation}
and
\begin{equation}
  \cos\rho = \frac{ (c-1)^2 \cos^2 \theta - \sin^2\theta }{(c-1)^2 \cos^2 \theta + \sin^2\theta } \;, 
\end{equation}
\begin{equation}
  \sin\rho = \frac{ 2(c-1) \cos\theta\sin\theta }{(c-1)^2 \cos^2 \theta + \sin^2\theta } \;, 
\end{equation}

Using $r^* = r^{-1}$, $t_+^* = t_-$, $t_- = t_+^{-1}$:
\begin{equation}
  \phi^*_y(\mu) = A^* \left[ t_-^y - r^{-1}t_+^y \right] \;,
\end{equation}
and so
\begin{equation}
  \phi_y(\mu) \phi^*_y(\mu) = - |A|^2 \left[ r t_-^y -2  + (rt_-^y)^{-1} \right] \;,
\end{equation}
which can be `simplified' to
\begin{equation}
  \phi_y(\mu) \phi^*_y(\mu) = 4|A|^2 \sin^2 \left( \frac{\rho - 2y\theta}{2} \right) \;.
\end{equation}

\subsection{Chebyshev Polynomial Solutions}

The recursion relation 
\begin{equation}
  \mu \phi_y  = \phi_{y+1} + \phi_{y-1} \;,
\end{equation}
can be satisfied by the functions $T_n(x)$ and  $U_n(x)$, which are the Chebyshev Polynomials of the first and second kind respectively, which satisfy
\begin{eqnarray}
  T_0(x) & = & 1 \nonumber \\
  T_1(x) & = & x \nonumber \\
  2xT_n(x) & = & T_{n+1}(x) + T_{n-1}(x) \;
\end{eqnarray}
and
\begin{eqnarray}
  U_0(x) & = & 1 \nonumber \\
  U_1(x) & = & 2x \nonumber \\
  2xU_n(x) & = & U_{n+1}(x) + U_{n-1}(x) \;.
\end{eqnarray}
It is useful to note that these functions are real (for real $x$). We find numerically that the eigenvectors of the transfer matrix are also real. 

We can satisfy the recursion relation for $\phi_y$ alongside the boundary condition $\mu\phi_0 = c\phi_1$ with the solution
\begin{equation}
  \phi_y(2x) = \alpha T_y(x) + \beta U_n(x) \;,
\end{equation}
where $2x = \mu$. If we now set $\phi_0 = 1$, we find $\beta = 1 - \alpha$, and using the boundary condition we find
\begin{equation}
  \phi_y(2x) = 2\left( \frac{c-1}{c}\right) T_y(x) +  \left(\frac{2-c}{c} \right) U_n(x) \;.
\end{equation}

We can try a similar solution for the left eigenfunctions $\psi_y$:
\begin{equation}
  \psi_y(2x) = \gamma T_y(x) + \delta U_y(x) \;.
\end{equation}
Using the boundary condition 
\begin{equation}
  2x\psi_1 = \frac{c}{2x} \psi_1 + \psi_2
\end{equation}
we find
\begin{equation}
  \delta = \frac{(c-2)}{2(1-c)} \gamma \;.
\end{equation}
Then, by setting $\psi_0 = 1$ and using $2x\psi_0 = \psi_1$ we find
\begin{equation}
  \gamma = 2(c-1)
\end{equation}
and
\begin{equation}
  \psi_y(2x) = 2(c-1) T_y(x) + (2-c) U_y(x) = c\phi_y(2x) \;.
\end{equation}
for $y>0$. For $y=0$, $\psi_0 = 1$ as we fixed earlier. (Note: according to the formula for $y>0$, $\psi_0 = 2 $ which is not correct.)

\subsection{Calculating $P(y)$ and Moments}

As noted previously
\begin{equation}
  P(y) = \frac{ \bra{y} T^L \ket{y} }{ Z } \;,
\end{equation}
where
\begin{equation}
  Z = \sum_{y'=0}^\infty \bra{y'} T^L \ket{y'} \;.
\end{equation}
To calculatate $P(y)$ we must compute the integral
\begin{equation}
  P(y) = \frac{ \bra{y} \displaystyle\int_{-2}^2 \D \mu \; \mu^L T^L \ket{\phi(\mu)} \braket{\psi(\mu)}{y} }{ Z }
\end{equation}
With the eigenvectors calculated above, we find that, for $y>0$, 
\begin{equation}
  P(y) = \frac{ 2^{L+1}}{c} \frac{1}{ Z }\displaystyle\int_{-1}^1 \D x \; x^L \left[ \psi_y(2x) \right]^2 \;, 
\end{equation}
where
\begin{eqnarray}
  \left[ \psi_y(2x) \right]^2 & = & 4(c-1)^2 \left[ T_y(x) \right] ^2 \nonumber \\
			      & + & 4(c-1)(2-c) T_y(x)U_y(x) \nonumber \\
			      & + & (2-c)^2 \left[ U_y(x) \right]^2 \;.
\end{eqnarray}
For $y=0$:
\begin{equation}
  P(0) = \frac{ 2^{L+1}}{ Z }\displaystyle\int_{-1}^1 \D x \; x^L = \; \frac{ 2^{L+1}}{ Z } \frac{(-1)^L+1}{L+1} \;.
\end{equation}
Thus
\begin{equation}
  P(0) = \begin{cases}
	    0 & : L \mbox{ odd} \\
	    \frac{ 2^{L+2}}{ Z(L+1) }& : L \mbox{ even}
         \end{cases} \;.
\end{equation}
This makes some sense because it's impossible to impose the periodic boundary condition ($\bra{y} \cdots \ket{y}$) if $L$ is odd. I have assumedthat $Z$ doesn't diverge; I don't know whether or not it diverges, but I have a feeling it does.

Using the identity (according to wikipedia)
\begin{equation}
  2T_n(x) = U_y(x) - U_{y-2}(x) \;,
\end{equation}
we can write
\begin{equation}
  \psi_y(2x) = U_y(x) + (1-c) U_{y-2}(x) \;.
\end{equation}


% APPENDICES ------------------------------------------------------------------------------------------------------------------------------------------------------------

\newpage
\appendix

\section{Chebyshev Polynomial Relations (From Wikipedia)}

\begin{equation}
  T_n(x) = \frac{1}{2} ( U_n(x) - U_{n-2}(x) )
\end{equation}
\begin{equation}
  T_{n+1}(x) = xT_n(x) - (1-x^2)U_{n-1}(x)
\end{equation}
\begin{equation}
  T_n(x) = U_n(x) - xU_{n-1}(x)
\end{equation}

\begin{eqnarray}
 T_0(x) = 1 & \;, &  U_{-1}(x) = 0 \nonumber \\
 T_{n+1}(x) & = & xT_n(x) - (1-x^2)U_{n-1}(x) \nonumber \\
 U_n(x) & = & xU_{n-1}(x) + T_n(x)
\end{eqnarray}

\begin{equation}
  T_n(x)^2 - T_{n-1}(x)T_{n+1}(x) = 1-x^2 > 0 \;, \quad \mbox{for} \quad -1 < x < 1
\end{equation}
\begin{equation}
  U_n(x)^2 - U_{n-1}(x)U_{n+1}(x) = 1 > 0
\end{equation}

\begin{equation}
  \int_{-1}^{1} \frac{T_n(y)}{(y-x) \sqrt{1-y^2} } \D x = \pi U_{n-1}(x)
\end{equation}
\begin{equation}
  \int_{-1}^{1} \frac{ \sqrt{1-y^2} U_{n-1}(y) }{ y-x} \D x = -\pi T_n(x)
\end{equation}

\begin{equation}
  \int_{-1}^{1} \frac{T_n(x)T_m(x)}{\sqrt{1-x^2}} \D x = 
    \begin{cases}
      0 & :n \ne m \\
      \pi & : n = m = 0 \\
      \pi/2 & : n = m \ne 0 \\
    \end{cases}
\end{equation}
\begin{equation}
  \int_{-1}^{1} U_n(x)U_m(x) \sqrt{1-x^2} \D x = 
    \begin{cases}
      0 & :n \ne m \\
      \pi/2 & : n = m \ne 0 \\
    \end{cases} 
\end{equation}



 

% \section{Vector Notes}
% 
% \subsection{Definitions}
% 
% $\ket{y}$:
% \begin{equation}
%   \ket{0} = \begin{pmatrix}1 \\ 0 \\ 0 \\ \vdots \\ \end{pmatrix} \;,\quad \ket{1} = \begin{pmatrix}0 \\ 1 \\ 0 \\ \vdots \\ \end{pmatrix} \;, \quad \ket{2} = \begin{pmatrix}0 \\ 0 \\ 1 \\ \vdots \\ \end{pmatrix} \;,\quad \mbox{etc.} \;.
% \end{equation}
% $\bra{y}$:
% \begin{equation}
%   \bra{0} = \begin{pmatrix}1 & 0 & 0 & \cdots \end{pmatrix} \;,\quad \bra{1} = \begin{pmatrix}0 & 1 & 0 & \cdots \end{pmatrix} \;, \quad \bra{2} = \begin{pmatrix}0 & 0 & 1 & \cdots \end{pmatrix} \;,\quad \mbox{etc.} \;.
% \end{equation}
% \begin{equation}
%   \ket{\phi(\mu)} = \sum_{y=0}^\infty \phi_y(\mu) \ket{y} \;.
% \end{equation}
% \begin{equation}
%   \bra{\phi(\mu)} = \sum_{y=0}^\infty \phi^*_y(\mu) \bra{y} \;.
% \end{equation}
% \begin{equation}
%   \bra{\psi(\mu)} = \sum_{y=0}^\infty \psi_y(\mu) \bra{y} \;.
% \end{equation}
% \begin{equation}
%   \ket{\psi(\mu)} = \sum_{y=0}^\infty \psi^*_y(\mu) \ket{y} \;.
% \end{equation}
% Matrix, $T$:
% \begin{equation}
%   T = T_{yy'} \ket{y}\bra{y'} \;.
% \end{equation}
% Transfer matrix, $T$:
% \begin{eqnarray}
%   T_{yy'} & = & \sum_{x}\mu_x \phi_y(\mu_x) \phi^*_{y'}(\mu_x) \nonumber \\
%   T & = & \sum_{\mu} \mu \ket{\phi(\mu)} \bra{\phi(\mu)}
% \end{eqnarray}
% $T^L$:
% \begin{eqnarray}
%   T^L_{yy'} & = & \sum_{x}\mu^L_x \phi_y(\mu_x) \phi^*_{y'}(\mu_x) \nonumber \\
%   T^L & = & \sum_{\mu} \mu^L \ket{\phi(\mu)} \bra{\phi(\mu)}
% \end{eqnarray}
% 
% 

\end{document}
