\documentclass[a4paper,10pt]{article}
%\documentclass[a4paper,10pt]{scrartcl}

% PACKAGES

\usepackage[utf8]{inputenc}
\usepackage{graphicx}
\usepackage{amsmath}
\usepackage{amssymb}

% SETTINGS

\setlength{\parindent}{1em}
\setlength{\parskip}{1em}

\hoffset=-50pt
\voffset=-20pt
\textwidth=440pt
\textheight=650pt

%defaults:
%\textheight = 592pt
%\textwidth = 390pt

% COMMANDS

\newcommand{\fref}[1]{Figure \ref{#1}}
\newcommand{\sref}[1]{Section \ref{#1}}

\newcommand{\bra}[1]{\langle #1 |}
\newcommand{\ket}[1]{| #1 \rangle}
\newcommand{\braket}[2]{\langle #1 | #2 \rangle}
\newcommand{\ketbra}[2]{| #1 \rangle \langle #2 |}

\newcommand{\Ai}{\mathrm{Ai}}

% TITLE

\title{Update From Weizmann \#2}
\author{Justin Whitehouse}
\date{\today}

\pdfinfo{%
  /Title    ()
  /Author   ()
  /Creator  ()
  /Producer ()
  /Subject  ()
  /Keywords ()
}

\begin{document}
\maketitle

%\abstract{Research in collaboration with: David Mukamel, Weizmann Institute of Science, Israel; Martin Evans and Richard Blythe, University of Edinburgh}

\tableofcontents
\newpage

\section{Introduction}

In the last few days I have been (numerically) looking at the heights distribution, in particular at $p = 0.5$. Richard recently presented an exactly solvable case for the model, which I tried to use to fit to the measured height distribution.

\subsection{Summary}

From Richard's solution I have derived a solution for the probability, $P(y)$, of a point on the interface having height $y$ in terms of the Airy function $\Ai(x)$ in essentially exactly these same way as we did previously for our mean-field theory. 

The solution I've found can be fit quite well to the measured distribution (with enough free paramters...). I think this corroborates our numerical evidence from the width measurements for a width exponent of $1/3$ near/at $p=0.5$.

Maybe in (the near?) future I will try to analyse the general case mean-field theory as perturbation/deviation from Richard's equilibrium solution (because the structure is very similar).

I've quickly recapped a couple of Richard's results which I use in my analysis before giving a brief overview of what I've done.

\subsection{Note about notation}

Richard has used $N$ for the system size. I will instead be using $L$, to be consistent with my own notation.

% RICHARD OVERVIEW ================================================================

\section{Quick Overview of Richard's Solution}

Richard wrote down a master equation for which a stationary solution can be found for the special parameter combination
\begin{equation}\label{eq:special_combo}
 \left( \frac{p}{1-p} \right)^{\frac{L}{2}} = \frac{1-u}{u} \;.
\end{equation}
To hold for arbitrary $u$ in the $L\to\infty$ limit, we must take
\begin{equation}\label{eq:p-a}
 p = \frac{1}{2} \left( 1 + \frac{a}{N} \right) \;, 
\end{equation}
and so \eqref{eq:special_combo} becomes
\begin{equation}\label{eq:a-u}
 \mathrm{e}^a = \frac{1-u}{u} \;.
\end{equation}

The stationary distribution $P(\mathcal{C},y_0)$ can be written as a matrix product
\begin{equation}
 P(\mathcal{C},y_0) = \frac{
 					\bra{y_0} \prod_{i=1}^L [\tau_i D + (1-\tau_i)E] \ket{y_0}
                                           }
                                           {
                                            Z
                                           }
\end{equation}
where 
\begin{equation}
 Z = \sum_{y_0=0}^\infty \bra{y_0} (D+E)^L \ket{y_0}
\end{equation}
and $\bra{y} \in \{ \bra{0}, \bra{1}, \ldots \}$ is the height basis ($\braket{y}{y'} = \delta_{y,y'}$). 
The representation of the matrices $D$ and $E$ in this basis is
\begin{equation}
 \bra{y'}D\ket{y} = \begin{cases}
                              \left( \frac{ 1-p}{p} \right)^{\frac{y-1}{2}} & y-y'=1 \\
                              0 & \mbox{otherwise} 
                             \end{cases}
\end{equation}
\begin{equation}
 \bra{y'}E\ket{y} = \begin{cases}
                              \left( \frac{ 1-p}{p} \right)^{\frac{y}{2}} & y-y'=-1 \\
                              0 & \mbox{otherwise} 
                             \end{cases}
\end{equation}                         

\section{Defining $P(y)$, Transfer Matrix, Eigenfunctions}

I have defined the probability that a site picked at random has height $y$ as
\begin{equation}
 P(y) = \sum_{\mathcal{C}} P(\mathcal{C},y) = \frac{\bra{y}  ( D + E)^L \ket{y}}{Z} \;.
\end{equation}
Defining
\begin{equation}
 \rho = \left( \frac{ 1-p}{p} \right) \;,
\end{equation}
the matrix $T = (D+E)$ has the structure
\begin{equation}
  T = \begin{pmatrix}
         0      & \rho^0  & 0      & 0     & 0      & \cdots \\
         \rho^0  & 0      & \rho^1  & 0     & 0      &        \\
         0      & \rho^1  & 0      & \rho^2 & 0      &        \\
         0      & 0      & \rho^2  & 0     & \rho^3  &        \\
       \vdots &        &         &        &         & \ddots \\
      \end{pmatrix} \;,
\end{equation}
which as Richard has already pointed out, is very similar to the transfer matrix $T'$ we have in the $p=1$ mean-field theory:
\begin{equation}
  T' = \begin{pmatrix}
         0      & c    & 0      & 0     & 0      & \cdots \\
         q^1  & 0      & q^1  & 0     & 0      &        \\
         0      & q^2  & 0      & q^2 & 0      &        \\
         0      & 0      & q^3  & 0     & q^3  &        \\
       \vdots &        &         &        &         & \ddots \\
      \end{pmatrix} \;.
\end{equation}

Similarly to how we did it for the mean-field theory, we can define the eigenvectors of $T$ as
\begin{equation}
 T\ket{\phi} = \mu\ket{\phi}
 \end{equation}
where
\begin{equation}
 \ket{\phi} = \sum_{y=0}^\infty \phi_y \ket{y} \;.
\end{equation}

\subsection{Eigenfunction}

Using the expressions above we can write down the recursion relations
\begin{equation}\label{eq:recursion_bc}
 \phi_2 = \mu \phi_0 \;,
\end{equation}
\begin{equation}\label{eq:recursion}
 \rho^{y-1} \phi_{y-1} + \rho^{y} \phi_{y+1} = \mu \phi_y \;, \quad y > 0 \; .
\end{equation}
Maybe we could find a solution from this for $\phi_y$ (perhaps in terms of Chebyshev polynomials, but I'm going to make a continuum approximation, as we did previously.

With $p = (1/2)(1 + a/L)$, we find that $\rho \simeq 1 - 2a/L$, so we can write $\rho = 1 - \epsilon$, with $\epsilon \ll 1$ (same as $q \simeq 1- \epsilon$ in earlier work). We make the expansion
\begin{equation}
 \phi_{y\pm1} \simeq \phi(y) \pm \phi'(y) + \frac{1}{2} \phi''(y) + \mbox{ h.o.t}\;,
\end{equation}
to find
\begin{equation}
 (3 - \mu - 2y\epsilon)\phi(y) + \phi''(y) = 0  \mbox{ ( $+$ h.o.t)}\;,
\end{equation}

This is essentially the same differential equation we found in our earlier mean-field theory. The boundary condition can only be satisfied with $\mu - 3 = \delta$, with $\delta \ll 1$. We map the equation onto Airy's equation, and find
\begin{equation}\label{eq:phi(y)}
 \phi(y) = c \; \Ai[ (y + b)(2\epsilon)^{1/3} + z_0] \;, \quad \Ai[z_0] = 0 \;,
\end{equation}
where $b$ and $c$ are constants.

\subsection{Probability Distribution}

From \eqref{eq:phi(y)} we can write
\begin{equation}
 P(y) = \frac{\Ai[ (y + b)(2\epsilon)^{1/3} + z_0] ^2}{Z} \;, 
\end{equation}
where
\begin{equation}
 Z = \int_0^\infty \mathrm{d}x \; \Ai[(x + b)(2\epsilon)^{1/3} + z_0] ^2 \;.
\end{equation}

\section{Fit to data}

We can rewrite $P(y)$ as
\begin{equation}\label{eq:P(y)}
 P(y) = \frac{(2\epsilon)^{1/3}\Ai[ (y + b)(2\epsilon)^{1/3} + z_0] ^2}{Z} \;, 
\end{equation}
where
\begin{equation}
 Z = \int_{b(2\epsilon)^{1/3} + z_0}^\infty \mathrm{d}z \; \Ai[z] ^2 \;.
\end{equation}
To see how this solution compares to data, I will (not very precisely or rigorously) fit this function to some height distribution data obtained for $p=0.5$, through the parameters $Z$ and $b$. By choosing $u$ and knowing the system size $L$ I can set $\epsilon = 2a/L$ and set $a$ by using $a = \ln( (1-u)/u)$.

Plotted in \fref{fig:Airy_2fit} we can see that the distribution in \eqref{eq:P(y)} has roughly the right shape, but a good fit cannot be obtained using $Z$ and $b$ alone. Fitting using $\epsilon$ as well allows us to obtain quite a good fit (\fref{fig:Airy_3fit}).

A quick observation: in \fref{fig:Airy_2fit} I used $\epsilon = 2a/L$, with $a = \ln((1-u)/u)$, $u=0.4$ and $L=1024$. This gives $\epsilon = 7.92\times10^{-4}$, which is approximately twice the value for which the fit was obtained. (it would be nice if it turns out I got a factor of two wrong somewhere and that we can fit this with just the two parameters $Z$ and $b$.)

A second quick observation: with $u=0.4$, $L = 1024$, the corresponding value of $p$ according to \eqref{eq:special_combo} is $p = 0.50020$.

\begin{figure}
 \centering
 \includegraphics[width=0.7\textwidth]{img/hd_l10p05u04_AiryZ06b-2.png}
 \caption{Attempt at a fit of the height distribution for $p=0.5$, $u = 0.4$ using $Z$ and $b$ (values chosen by eye).}
 \label{fig:Airy_2fit}
\end{figure}

\begin{figure}
 \centering
 \includegraphics[width=0.7\textwidth]{img/hd_l10p05u04_AiryZ05b1e00004.png}
 \caption{Attempt at a fit of the height distribution for $p=0.5$, $u=0.4$ using $Z$, $b$ and $\epsilon$ (values chosen by eye).}
 \label{fig:Airy_3fit}
\end{figure}

\end{document}
